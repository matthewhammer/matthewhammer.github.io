\documentclass[10pt,letterpaper]{article}

\usepackage{hyperref}
\usepackage{geometry}
\usepackage{charter}

\newcommand{\BeanCounter}[1]{(Acceptance~Rate:~#1\%)}

% Fonts
%\usepackage[T1]{fontenc}
%\usepackage[urw-garamond]{mathdesign}

% Set your name here
\def\name{Matthew A. Hammer}

% The following metadata will show up in the PDF properties
\hypersetup{
  colorlinks = true,
  urlcolor = blue,
  pdfauthor = {\name},
  pdftitle = {\name: Curriculum Vitae},
  pdfsubject = {Curriculum Vitae},
  pdfpagemode = UseNone
}

\geometry{
  body={6.5in, 9.0in},
  left=1.0in,
  top=1.0in
}

% Customize page headers
\pagestyle{myheadings}
\markright{\name}
\thispagestyle{empty}

% Custom section fonts
\usepackage{sectsty}
\sectionfont{\rmfamily\mdseries\Large}
\subsectionfont{\rmfamily\mdseries\itshape\large}

% Other possible font commands include:
% \ttfamily for teletype,
% \sffamily for sans serif,
% \bfseries for bold,
% \scshape for small caps,
% \normalsize, \large, \Large, \LARGE sizes.

% Don't indent paragraphs.
\setlength\parindent{0em}

% Make lists without bullets and compact spacing
\renewenvironment{itemize}{
  \begin{list}{}{
    \setlength{\leftmargin}{1.5em}
    \setlength{\itemsep}{0.25em}
    \setlength{\parskip}{0pt}
    \setlength{\parsep}{0.25em}
  }
}{
  \end{list}
}

\newenvironment{resexperience}[4]{
\item #1 (#3--#4) \\
\emph{#2}
}{
}


\begin{document}

% Place name at left
{{\huge \name}
\\
Assistant Professor
}

% Alternatively, print name centered and bold:
%\centerline{\huge \bf \name}

\bigskip

\begin{minipage}[t]{0.495\textwidth}
Computer Science Department\\
University of Colorado\\
1111 Engineering Dr, Boulder, CO 80309
\end{minipage}
\begin{minipage}[t]{0.495\textwidth}
  Email: \href{mailto:matthew.hammer@colorado.edu}{Matthew.Hammer@Colorado.Edu} \\
%  Homepage: \href{http://cs.colorado.edu/~maha2973}{http://cs.colorado.edu/{\textasciitilde}maha2973}
  Homepage: \href{http://matthewhammer.org}{http://matthewhammer.org}
\end{minipage}

\section*{Education}

\begin{tabular}[t]{@{}l l l@{}}
PhD (Computer Science) & University of Chicago & 2012 \\
MS  (Computer Science) & Toyota Technological Institute at Chicago & 2007 \\
BS  (Computer Science) & University of Wisconsin & 2005 \\
\end{tabular}

%% \begin{itemize}
%%   \item Ph.D. Computer Science, University of Chicago, 2012.

%%   \item M.S. Computer Science, Toyota Technological Institute at Chicago, 2007.

%%   \item B.S. Computer Science, University of Wisconsin, 2005.
%% \end{itemize}

\section*{Publications}
\begin{itemize}

\item
\href{http://hazelgrove.org/docs/hazel-vision-tr.pdf}{\textit{Toward a Semantics for Program Editors}}
\\
Cyrus Omar, Ian Voysey, Michael Hilton, Joshua Sunshine, Claire Le Goues, Jonathan Aldrich, \underline{Matthew A. Hammer}.
\\
The 2nd Summit on Advances in Programming Languages (\textbf{SNAPL 2017}).
\\
Monterey, California. May 2017.

\item
\href{https://arxiv.org/abs/1607.04180}{\textit{Hazelnut: A Bidirectionally Typed Structure Editor Calculus}}
\\
Cyrus Omar, Ian Voysey, Michael Hilton, Jonathan Aldrich, \underline{Matthew A. Hammer}.
\\
Principles of Programming Languages (\textbf{POPL 2017}).
\\
Paris, France. January 2017.
\\
\BeanCounter{27}

\item
\href{https://arxiv.org/abs/1608.06012}{\textit{A Vision for Online Verification-Validation}}
\\
\underline{Matthew A. Hammer}, Bor-Yuh Evan Chang, David Van Horn
\\
Generative Programming: Concepts \& Experience (\textbf{GPCE 2016}).
\\
Amsterdam, Netherlands. October 2016.
\\
\BeanCounter{32}

\item
\href{https://arxiv.org/abs/1608.06009}{\textit{The Random Access Zipper: Simple, Purely-Functional Sequences}}
\\
Kyle Headley, \underline{Matthew A. Hammer}.
\\
Trends in Functional Programming (\textbf{TFP 2016}).
\\
College Park, Maryland. June 2016. 

\item
\href{http://arxiv.org/abs/1503.07792}
{\textit{Incremental Computation with Names}}
\\
\underline{Matthew A. Hammer}, Joshua Dunfield, Kyle Headley, Nicholas Labich,
Jeffrey S. Foster and Michael Hicks.
\\
Object-Oriented Programming, Systems, Languages, and Applications
(\textbf{OOPSLA 2015}).
\\
Pittsburgh, USA. October 2015.
\\
\BeanCounter{25}

\item
\href{http://www.cs.umd.edu/~hammer/adapton}
{\textit{\textsc{Adapton}: Composable, Demand-driven Incremental Computation}}
\\
\underline{Matthew A. Hammer}, Yit Phang Khoo, Michael Hicks and Jeffrey S. Foster.
\\
Programming Language Design and Implementation (\textbf{PLDI 2014}).
\\
Edinburgh, Scotland. June 2014.
\\
\BeanCounter{20}

\item
\href{http://www.cs.umd.edu/~hammer/oakland2014}
{\textit{\textsc{Wysteria}: A Programming Language for Generic, Mixed-Mode Multiparty Computations}}
\\
Aseem Rastogi, \underline{Matthew A. Hammer} and Michael Hicks.
\\
35th IEEE Symposium on Security and Privacy (\textbf{IEEE~S\&P~2014})
\\
San Jose, California USA. May 2014.
\\
\BeanCounter{13.6}

\item
\href{http://www.cs.umd.edu/~hammer/jfp2014}
{\textit{Implicit Self-Adjusting Computation for Purely Functional Programs}}
\\
Yan Chen, Joshua Dunfield, \underline{Matthew A. Hammer} and Umut A. Acar.
\\
Journal of Functional Programming 2014 (\textbf{JFP 2014}).

\item
\href{http://www.cs.umd.edu/~hammer/plas2013}
{\textit{Knowledge Inference for Optimizing Secure Multi-party Computation}}
\\
Aseem Rastogi, Piotr Mardziel, \underline{Matthew A. Hammer} and Michael Hicks.
\\
Programming Languages and Analysis for Security (\textbf{PLAS 2013}).
\\
Seattle, Washington USA. June 2013.

\item
\href{http://www.cs.umd.edu/~hammer/oopsla11}
{\textit{Self-Adjusting Stack Machines}}
\\
\underline{Matthew A. Hammer}, Georg Neis, Yan Chen and Umut A. Acar 
\\
Object-Oriented Programming, Systems, Languages, and Applications
(\textbf{OOPSLA 2011}).
\\
Portland, Oregon USA. October 2011.
\\
\BeanCounter{23}

\item
\href{http://www.cs.umd.edu/~hammer/icfp11}
{\textit{Implicit Self-Adjusting Computation for Purely Functional Programs}}
\\
Yan Chen, Joshua Dunfield, \underline{Matthew A. Hammer} and Umut A. Acar 
\\
International Conference on Functional Programming (\textbf{ICFP 2011}).
\\
Tokyo, Japan. September 2011
\\
\BeanCounter{31}

\item
\href{http://www.cs.umd.edu/~hammer/pldi09}
{\textit{\textsc{Ceal}: A C-Based Language for Self-Adjusting Computation}}
\\
\underline{Matthew A. Hammer}, Umut A. Acar and Yan Chen.
\\
Programming Language Design and Implementation (\textbf{PLDI 2009}).
\\
Dublin, Ireland. June 2009.
\\
\BeanCounter{20}

\item
\href{http://www.cs.umd.edu/~hammer/ismm08}
{\textit{Memory Management for Self-Adjusting Computation}}
\\
\underline{Matthew A. Hammer} and Umut A. Acar.
\\
International Symposium on Memory Management (\textbf{ISMM 2008}).
\\
Tuscon, Arizona. June 2008.
\\
\BeanCounter{43}

\item
\href{http://www.cs.umd.edu/~hammer/papers/damp07.pdf}
{\textit{A Proposal for Parallel Self-Adjusting Computation}}
\\
\underline{Matthew Hammer}, Umut A. Acar, Mohan Rajagopalan, Anwar Ghuloum
\\
Workshop on Declarative Aspects of Multicore Programming (\textbf{DAMP 2007}).
\\
Nice, France. January 2007.

\item
\href{http://www.cs.umd.edu/~hammer/papers/ibmsj06quake.pdf}
{\textit{Running Quake II on a grid}}
\\
G. Deen, \underline{M. Hammer}, J. Bethencourt, I. Eiron, J. Thomas, and J. H. Kaufman.
\\
IBM Systems Journal 2006.

\end{itemize}

\section*{Theses}

\begin{itemize}
\item
\href{http://www.cs.umd.edu/~hammer/selfadjmachines}
{\textit{Self-Adjusting Machines}}
%\\
%Matthew A. Hammer
\\
University of Chicago, December 2012.
\\
Committee:
\begin{itemize}
\item John Reppy (\textit{Chair})
\item Umut A. Acar (\textit{PhD Advisor})
\item David MacQueen
\item Rupak Majumdar
\end{itemize}
\end{itemize}

\section*{Patents}

\begin{itemize}

%% \item 
%%   Viktors Berstis, John Bethencourt, Kevin Damm, Glenn Deen, Matthew A. Hammer, James H Kaufman, Toby Lehman
%%   \\
%%   \textit{Method for distributing and geographically load balancing location aware communication device client-proxy applications}
%%   \\
%%   \textbf{US Patent 7,702,784}
%%   %US Patent 7,428,588
  
\item
\textit{Distributing and geographically load balancing location aware communication device client-proxy applications}
\\
  Viktors Berstis, John Bethencourt, Kevin Damm, Glenn Deen, Matthew A. Hammer, James H Kaufman, Toby Lehman
\\
\textbf{US Patent 7,702,784}
%US Patent 7,428,588  ???

\item
\textit{Handling of players and objects in massive multi-player on-line games}
\\
  Viktors Berstis, John Bethencourt, Kevin Damm, Glenn Deen, Matthew A. Hammer, James H Kaufman, Toby Lehman
\\
\textbf{US Patent 8,057,307}

\item
\textit{Concurrent Management of Adaptive Programs}
\\
Matthew Hammer, Mohan Rajagopalan, Anwar Ghuloum
\\
\textbf{US Patent App. 11/750,441}

\end{itemize}

\section*{Funding}
\begin{itemize}
\item \href{https://www.nsf.gov/awardsearch/showAward?AWD_ID=1619282}
  {NSF Small: Online Verification-Validation (\$310k to CU Boulder)}
\item Mozilla Research Funding (unrestricted gift, \$90k)
\end{itemize}

\section*{Students}

\begin{itemize}
\item Kyle Headley (PhD program, CU Boulder)
\item Jared Wright (PhD program, CU Boulder)
\item Monal Narasimhamurthy (PhD program, CU Boulder)
\end{itemize}

\section*{Service}

\begin{itemize}

\item \textbf{Thesis Committees}:
  Max Russek (CU Undergrad; Spring 2016)

\item \textbf{SRC Judge}:
  PLDI 2016 Student Research Competition.
  
\item \textbf{Program Committee (PC) member}:
  GPCE 2017,
  ESOP 2017,
  PLAS 2015

\item \textbf{External Review Committee (ERC) member}: PLDI 2015

\item \textbf{External reviewer}:
ESOP 2017,  
ESOP 2016,  
IEEE~S\&P~2015,
POPL~2015, % review: #159C
OOPSLA~2014, % submissions: i3ql, 187 (incremental object queries)
PLAS~2014, % submission 4
SOFSEM~2014, % submission 101
PLDI~2013, % submission 106: ``Simple and efficient higher-order reactive programming''
POPL~2012, 
ICFP~2010, 
ML~Workshop~2009, 
PLDI~2008.

\item \textbf{Graduate Student Representative}.  
  May 2010--October 2011.
  \\
  Max Planck Institute for Software Systems.

  
\end{itemize}

\section*{Teaching}

\begin{itemize}

\item {
  \href{http://matthewhammer.org/courses/csci7000-s17}
  {\textbf{CSCI 7000}: \textbf{Programming languages for incremental computing}}
  \\
  University of Colorado, Boulder.  Spring 2017.  
}

\item {
  \href{http://matthewhammer.org/courses/csci5535-f16}
  {\textbf{CSCI 5535}: \textbf{Foundations of Programming Languages}}
  \\
  University of Colorado, Boulder.  Fall 2016. 
}

\item {
  \textbf{CSCI 7000}: \textbf{Programming language design for interfaction}
  \\
  University of Colorado, Boulder.  Spring 2016.  
}

\item {
  \href{https://www.cs.colorado.edu/~maha2973/csci5535/f15/}
  {\textbf{CSCI 5535}: \textbf{Foundations of Programming Languages}}
  \\
  University of Colorado, Boulder.  Fall 2015.  
}

  
\item {
  \textbf{CMSC 631}: \textit{Program Analysis and Understanding}.
  \\
  University of Maryland, College Park.  Spring 2013.
  \\
  Co-instructed with Michael Hicks, Jeffrey S. Foster and Stevie Strickland.
}

\item { 
  \textbf{Teaching assistant for CMCS 336}: \textit{Type Systems for Programming Languages}.
  \\
  Toyota Technological Institute / University of Chicago.  Winter 2008.
  \\
  %
  Instructors: Umut Acar and Amal Ahmed.
}
\end{itemize}


\section*{Talks}

\begin{itemize}

\item
\href{https://arxiv.org/abs/1608.06012}{\textit{A Vision for Online Verification-Validation}}
\\
\underline{Matthew A. Hammer}, Bor-Yuh Evan Chang, David Van Horn
\\
Generative Programming: Concepts \& Experience (\textbf{GPCE 2016}).
\\
Amsterdam, Netherlands. October 2016.

\item
\href{http://arxiv.org/abs/1503.07792}
{\textit{Incremental Computation with Names}}
\\
\underline{Matthew A. Hammer}, Joshua Dunfield, Kyle Headley, Nicholas Labich,
Jeffrey S. Foster and Michael Hicks.
\\
Object-Oriented Programming, Systems, Languages, and Applications
(\textbf{OOPSLA 2015}).
\\
Pittsburgh, USA. October 2015.

\item
{\textit{\textsc{Wysteria}: A Programming Language for Generic, Mixed-Mode Multiparty Computations}}
\\
\href{http://www.dagstuhl.de/en/program/calendar/semhp/?semnr=14492}
{\textbf{Dagstuhl seminar 14492}: The synergy between programming languages and cryptography}.
\\
Schloss Dagstuhl. Wadern, Germany. December 2014.

\item
{\textit{\textsc{Adapton}: Composable, Demand-driven Incremental Computation}}
\\
Programming Language Design and Implementation (\textbf{PLDI 2014}).
\\
Edinbugh, Scotland. June 2014.

\item \textit{Self-Adjusting Stack Machines}
\\
Object-Oriented Programming, Systems, Languages, and Applications
(\textbf{OOPSLA 2011}).
\\
Portland, Oregon USA. October 2011.

\item \textit{Self-Adjusting Stack Machines and the CEAL Compiler}
\\
Invited talk.  Max Planck Institute for Software Systems
advisory board visit day.
\\
Frankenstein, Rhineland-Palatinate Germany. May 2011.

\item \textit{A Compilation Framework for Self-Adjusting Computation}
\\
Dissertation proposal.
\\
Chicago, Illinois USA. December 2010.

\item \textit{\textsc{Ceal}: A C-Based Language for Self-Adjusting Computation}
\\
Programming Language Design and Implementation (\textbf{PLDI 2009}).
\\
Dublin, Ireland. June 2009.


\item \textit{Memory Management for Self-Adjusting Computation},
\\
International Symposium on Memory Management (\textbf{ISMM 2008}).
\\
Tuscon, Arizona. June 2008.

\item \textit{A Proposal for Parallel Self-Adjusting Computation}, 
\\
Workshop on Declarative Aspects of Multicore Programming (\textbf{DAMP 2007}).
\\
Nice, France. January 2007.

\end{itemize}


\section*{Software}

\begin{itemize}
\item \href{http://adapton.org}{\textsc{Adapton}: Composable, Demand-Driven Incremental Computation}.
  %   
  \textsc{Adapton} provides library primitives (currently in OCaml and Rust, and previously, in Python)
  for creating incremental computation (IC).  Unlike prior approaches,
  \textsc{Adapton} supports demand-driven IC (e.g., computations that use
  laziness).

\item \href{https://bitbucket.org/aseemr/wysteria/wiki/Home}{\textsc{Wysteria}: A Programming Language for Generic, Mixed-mode Multiparty Computation}.
  %   
  \textsc{Wysteria} is a high-level functional programming language for writing mixed-mode secure computations. Such computations interleave local, private computations with secure multiparty computations.

\item \href{http://ceal.mpi-sws.org}{\textsc{Ceal}: A C-based language (compiler and run-time system) for self-adjusting computation}.
  % 
  \textsc{Ceal} extends C with a small set of primitives that allow
  programmers to write self-adjusting computations in a manner similar
  to conventional C programming.
\end{itemize}


\section*{Student Internships}

\begin{itemize}     
 
  \begin{resexperience}
    {Intel, Programming Systems Lab at Santa Clara}
    {Graduate Research Intern}
    {June 2007}
    {September 2007}    
  \end{resexperience}

  \begin{resexperience}
    {Intel, Programming Systems Lab at Santa Clara}
    {Graduate Research Intern}
    {June 2006}
    {September 2006}    
  \end{resexperience}

  \begin{resexperience}
    {IBM, Almaden Research Center}
    {Research Intern}
    {May 2005}
    {September 2005}
    
%     \htmladdnormallink[hc] {Healthcare information
%       management}{http://www.almaden.ibm.com/software/hc/} projects,
%     including \htmladdnormallink[stem]{Spatiotemporal
%       Epidemiological Modeler
%       (STEM)}{http://www.alphaworks.ibm.com/tech/stem} and SynthEHR.
%     STEM uses a differential model and graphics to create maps
%     that model national and world health epidemics and help public
%     administrators create public health policy.  SynthEHR is a
%     distributed, agent-based simulation used to generate synthetic
%     health records for related research projects.  Contact: James
%     Kaufman \emph{kaufman@almaden.ibm.com}

  \end{resexperience}
  
  \begin{resexperience}
    {IBM, Almaden Research Center}
    {Research Intern}
    {May 2004}
    {August 2004} 

%     Worked in IBM's research division on both
%     grid computing middleware and a system for simulating the
%     spread of infectious diseases.  The grid system's focus was on
%     load-balancing a running interactive application, such as an
%     online game.  The simulation used GIS data for both model
%     creation and visualization.  Contact: James Kaufman
%     \emph{kaufman@almaden.ibm.com}

  \end{resexperience}    
  
  \begin{resexperience}
    {IBM, Extreme Blue Program}
    {Computer Science Intern}
    {June 2003}
    {August 2003} 

%     Worked on \emph{GameGrid}, a application of IBM grid technology to
%     Massively Multiplayer Online Games (MMOGs).  Focused on design and
%     development of a system deploying Quake II as a proof of concept,
%     grid-enabled game.  The project was featured on Slashdot,
%     eWeek.com, and Technology Review and demonstrated at LinuxWorld
%     2003. Contact: James Kaufman \emph{kaufman@almaden.ibm.com}

  \end{resexperience}
  
%   \begin{resexperience}
%     {\htmladdnormallink[upl]{Undergraduate Projects Lab (UPL)}
%       {http://www.upl.cs.wisc.edu/}} {Coordinator}{November
%       2002}{May 2005} Helped with system administration of user
%     accounts and network systems as well as non-technical
%     administrative tasks.  Coordinated extra-curricular software
%     projects both alone and with other students purely for the joy
%     of creation and cooperation. Contact: UPL Coordinators
%     \emph{upl@upl.cs.wisc.edu}
%   \end{resexperience}

\end{itemize}



\medskip

% Footer
\begin{center}
  \begin{small}
    Last updated: \today
  \end{small}
\end{center}

\end{document}
