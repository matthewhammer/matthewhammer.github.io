\documentclass[11pt]{article}
\usepackage{fullpage}
\usepackage{latexsym}
\usepackage{verbatim}
\usepackage{code,proof,amsthm,amssymb, amsmath}
\usepackage{ifthen}
\usepackage{graphics}
\usepackage{xspace}
\usepackage{hyperref}
\usepackage{mathpartir}

%% Question macros
\newcounter{question}[section]
\newcounter{extracredit}[section]
\newcounter{totalPoints}
\setcounter{totalPoints}{0}
\newcommand{\question}[1]
{
\bigskip
\addtocounter{question}{1}
\addtocounter{totalPoints}{#1}
\noindent
{\textbf{Task \thesection.\thequestion}}[#1 points]:


}
\newcommand{\ecquestion}
{
\bigskip
\addtocounter{extracredit}{1}
\noindent
\textbf{Extra Credit \thesection.\theextracredit}:}

%% Set to "false" to generate the problem set, set to "true" to generate the solution set
\def\issolution{false}

\newcounter{taskcounter}
\newcounter{taskPercentCounter}
\newcounter{taskcounterSection}
\setcounter{taskcounter}{1}
\setcounter{taskPercentCounter}{0}
\setcounter{taskcounterSection}{\value{section}}
\newcommand{\mayresettaskcounter}{\ifthenelse{\value{taskcounterSection} < \value{section}}
{\setcounter{taskcounterSection}{\value{section}}\setcounter{taskcounter}{1}}
{}}

% Solution-only - uses an "input" so that it's still safe to publish the problem set file
\definecolor{solutioncolor}{rgb}{0.0, 0.0, 0.5}
\newcommand{\solution}[1]
  {\ifthenelse{\equal{\issolution}{true}}
  {\begin{quote}
    \addtocounter{taskcounter}{-1}
    \fbox{\textcolor{solutioncolor}{\bf Solution \arabic{section}.\arabic{taskcounter}}}
    \addtocounter{taskcounter}{1}
    \textcolor{solutioncolor}{\input{./solution/#1}}
  \end{quote}}
  {}}

\newcommand{\qbox}{\fbox{???}}

\setcounter{taskcounter}{1}
\setcounter{taskPercentCounter}{0}
%\setcounter{taskcounterSection}{\value{section}}

\newcommand{\ttt}[1]{\texttt{#1}}

%% part of a problem
\newcommand{\task}[1]
  {\bigskip \noindent {\bf Task\addtocounter{taskPercentCounter}{#1} \arabic{taskcounter}\addtocounter{taskcounter}{1}} (#1 pts).}

\newcommand{\ectask}
  {\bigskip \noindent {\bf Task \arabic{section}.\arabic{taskcounter}\addtocounter{taskcounter}{1}} (Extra Credit).}

\newcommand{\val}[1]{#1~\textsf{val}}
\newcommand{\num}[1]{\texttt{num}[#1]}
\newcommand{\str}[1]{\texttt{str}[#1]}
\newcommand{\plus}[2]{\texttt{plus}(#1; #2)}
\newcommand{\mult}[2]{\texttt{times}(#1; #2)}
\newcommand{\cat}[2]{\texttt{cat}(#1; #2)}
\newcommand{\len}[1]{\texttt{len}(#1)}
\newcommand{\abst}[2]{#1.#2}
\newcommand{\letbind}[3]{\texttt{let}(#1; \abst{#2}{#3})}
\newcommand{\steps}[2]{#1 \mapsto #2}
\newcommand{\subst}[3]{[#1/#2]#3}
\newcommand{\err}[1]{#1~\textsf{err}}

\newcommand{\proves}{\vdash}
\newcommand{\hasType}[2]{#1 : #2}
\newcommand{\typeJ}[3]{#1 \proves \hasType{#2}{#3}}
\newcommand{\ctx}{\Gamma}
\newcommand{\emptyCtx}{\emptyset}
\newcommand{\xCtx}[2]{\ctx, \hasType{#1}{#2}}
\newcommand{\typeJC}[2]{\typeJ{\ctx}{#1}{#2}}
\newcommand{\numt}{\texttt{num}}
\newcommand{\strt}{\texttt{str}}

\newcommand{\laz}[1]{\left[ #1 \right]}

\newcommand{\E}{\textbf{\textsf{E}}\xspace}
\newcommand{\T}{\textbf{\textsf{T}}\xspace}
\newcommand{\ETSP}{\textbf{\textsf{ETSP}}\xspace}
\newcommand{\ETSPL}{\textbf{\textsf{ETSPL}}\xspace}

\newcommand{\typ}[0]{\tau}
\newcommand{\List}[1]{\textsf{List}(#1)}
\newcommand{\Nil}[0]{\textsf{Nil}}
\newcommand{\Cons}[2]{\textsf{Cons}(#1, #2)}
\newcommand{\match}[3]{\textsf{match}(#1, #2, #3)}
\newcommand{\fold}[3]{\textsf{fold}(#1, #2, #3)}
\newcommand{\map}[2]{\textsf{map}(#1, #2)}

\title{Assignment \#3: \\
  %Abstract Binding Trees,
  Meta Theory and Implementation:
  \\Language~\ETSP, extended with generic lists~\textbf{\textsf{L}}}

\author{Fundamentals of Programming Languages}
% put together by Cyrus Omar and Shayak Sen, based on assignments from S10 and S09
\date{Out: Thursday, Oct 13th, 2016\\
      Due: Thursday, Oct 27th, 2016 11:59pm EST}

\begin{document}
\maketitle

The tasks in this homework ask you to prove (``meta theoretical'')
properties about the language~\ETSPL, which combines languages~\E, \T
with sums and products; in homework \#2, we considered a large subset
of this language (everything except language~\E).  Building on these
language fragments, we will add generic lists, which like natural
numbers, consist of a recursive structure.
%
We call the resulting language~\ETSPL.
%
Unlike natural numbers, these generic list structures carry data of an
arbitrary (generic) \emph{element type}.

This homework also asks you to program an implementation of this
combined language in OCaml.
%
\emph{We did this together in class for fragments of this language,
  but did not complete the implementation in OCaml; use this code and
  these videos as a guide}.

\paragraph{Grading criteria for proofs:} 
To receive full credit for any proof below, you must \emph{at least} do the following:
\begin{itemize}
\item At the beginning of your proof, specify over what structure or
  derivation you are performing induction (i.e., which structure's
  inductive principle are you using?)
\item In the inductive cases of the proof, specify how you are applying the inductive hypothesis, and what result it gives you.
\end{itemize}
\textbf{If you omit these steps and/or do not make them explicit, you
  will receive zero credit for your proof.}  If you attempt to do
these steps, but you make a mistake, you may still receive some partial credit, depending on your proof.

\paragraph{Hint:}
If you are unsure about how to structure these proofs to receive full credit, 
\textbf{please refer to the HW \#1 and \#2 Solutions} on Moodle as a reference and guide.

\paragraph{Note on omitting redundant proof cases:} In the proofs below, some cases are very
similar to other cases, e.g., the cases for \texttt{plus} and
\texttt{times} in the proofs below are likely to be analogous, in that
(nearly) the same proof steps are used in each.
%
When this happens, you can omit the redundant cases as follows: If you
do one case, say for \texttt{plus}, you may (optionally) write in the
other case for \texttt{times} that it is ``analogous to the case
above, for \texttt{plus}''.  You must make this omission explicit, to
show that you have thought about it.  Further, this shortcut is only
applicable when the cases really are analogous, and (nearly) the same
steps apply in the proof.  \textbf{When in doubt, do not omit the
  proof case.}

\section*{Tasks}

\paragraph{Syntax.} The syntax of generic lists consists of several additional forms:

\[
\begin{array}{lllll}
Types &
\typ & ::= & \cdots
\\
    && | & \List{\typ}
\\[2mm]
Expressions &
e & ::= & \cdots 
\\
 &&|& \textsf{Nil}
\\
 &&|&      \textsf{Cons}(e_1, e_2)
\\
 &&|& \textsf{match}(e_l, e_n, x.y.e_c)
\\
 &&|& \textsf{fold}(e_l, e_n, x.y.e_c)
\\
 &&|& \textsf{map}(e_l, x.e_h)
\end{array}
\]

\paragraph{Statics} There is a new typing rule for each new syntactic form:

\begin{mathpar}
\inferrule[nil]{  }{
\Gamma \vdash \Nil : \List{\typ}
}
\and
\inferrule[cons]{
\Gamma \vdash e_1 : \tau 
\\\\
\Gamma \vdash e_2 : \List{\typ}
}
{\Gamma \vdash \Cons{e_1}{e_2} : \List{\typ}}
\and
\inferrule[match]{
\Gamma \vdash e_l : \List{\typ_1}
\\\\
\Gamma \vdash e_n : \typ
\\\\
\Gamma, x:\typ_1, y:\List{\typ_1} \vdash e_c : \typ
}
{
\Gamma \vdash \match{e_l}{e_n}{x.y.e_c} : \typ
}
%
\and
\inferrule[fold]{
\Gamma \vdash e_l : \List{\typ_1}
\\\\
\Gamma \vdash e_n : \typ
\\\\
\Gamma, x:\List{\typ_1}, y: \typ \vdash e_c : \typ
}
{
\Gamma \vdash \fold{e_l}{e_n}{x.y.e_c} : \typ
}
%
\and
\inferrule[map]{
\Gamma  \vdash e_l : \List{\typ_1}
\\\\
\Gamma, x:\typ_1 \vdash e_h : \typ_2
}
{
\Gamma \vdash \map{e_l}{x.e_h} : \List{\typ_2}
}
\end{mathpar}

\task{20} Dynamics for Language~\ETSPL.

\begin{enumerate} 
\item Add rules for the $e~\textsf{val}$ judgement.
\item Add dynamics rules for stepping these forms.
\end{enumerate}

\task{20} Meta theory for Language~\ETSPL.

\begin{enumerate} 
\item State the substitution lemma for Language~\ETSPL.
\item State the canonical forms lemma for Language~\ETSPL.
\item State and prove progress for Language~\ETSPL.
\item State and prove preservation for Language~\ETSPL.
\end{enumerate}

You only need to do the \emph{new} cases of the proofs, for generic lists.

\paragraph{\underline{Hint (repeated again, for emphasis)}:}
If you are unsure about how to structure these proofs to receive full credit, 
\textbf{please refer to the HW \#1 and \#2 Solutions} on Moodle as a reference and guide.

\task{20} Implement the theory of Language~\ETSPL in OCaml.

\task{20} Implement tests for each construct in the language (at least
one test per construct; some tests can test multiple constructs).
%
In particular, implement a function~\texttt{test\_pap} that tests
\emph{\underline{p}rogress \underline{a}nd \underline{p}reservation}.  Given an expression, this function
computes a type for the expression, then steps that expression until
it is a value.  After each step, it computes a new type for the
expression and asserts that the new and old type are equal.
%
The function returns the final expression (a value) when it
terminates.
%
For each of your tests, use \texttt{test\_pap} to test that progress
and preservation indeed hold on your initial expression, in addition
to testing that the final value matches the one that you expect.

\paragraph{How to implement a Language Theory:}
When we say ``implement Language $X$ in OCaml'', we mean precisely the
following.  For a concrete example, see our implementation of
Language~\E as a guide, where we did this together in class.  (There
is a lecture video and OCaml code from September 29 available online).
\begin{enumerate}
\item Define syntax forms as OCaml datatypes 
\begin{enumerate}
\item Define variables~\texttt{var} as OCaml strings (type~\texttt{string})
\item Define the syntax of expressions as a new OCaml datatype named~\texttt{exp}
\item Define the syntax of types as a new OCaml datatype named~\texttt{typ}
\item Define type contexts~\texttt{gamma} as OCaml lists of variable-type pairs.
\item Implement pretty-printing functions for expressions, types and contexts:\\ 
\begin{tabular}{ll}
\texttt{exp\_string} :& \texttt{exp -> string}
\\
\texttt{typ\_string} :& \texttt{typ -> string}
\\
\texttt{gam\_string} :& \texttt{gamma -> string}
\end{tabular}
\end{enumerate}
\item Implement a function \texttt{is\_val : exp -> bool} that implements a check for the $e~\textsf{val}$ judgment
\item Implement a substitution function \texttt{subst : exp -> var -> exp -> exp}.  
{
Make sure that you implement \emph{shadowing} correctly, and you do
not allow \emph{variable capture}.  
%
See our implemention of the \texttt{Let} case in language~\E
as a reference; notice how we compare the \texttt{Let}-bound variable
against the one being substituted, and do not substitute further if
they are the same.  }
\item Implement a type-checking function \texttt{exp\_typ : gamma ->
  exp -> typ option}.  

  \paragraph{Hint:}~
  Note that the $\Nil$ form has a similar problem to lambda and some
  other forms in the prior homework: The list type is not clear from
  the term $\Nil$, and thus could be anything.
  % 
  To give programs enough information to type them, add a type
  annotation to the $\Nil$ form, and other forms, but only \emph{if they
    require it}. 

%\item Implement a suite of five \textbf{interesting} tests of type-checking.  
%(``Interesting'' means that you attempt to cover different execution paths of your implementation with each test).
%%

\item Implement a steps-to function \texttt{step : exp -> exp}. It
  should take exactly one small step, or raise an exception if the
  expression is a value.

\item Implement a multiple-steps function \texttt{test\_pap : exp ->
  exp}. It should take as many steps as possible, and it should test
  progress and preservation as it steps.  This is precisely what you
  proved in your \emph{meta theory} proofs about well-typed programs
  in the language.  
  
  Use the pretty-printing functions above to print the type, and to
  print the expression as it steps.  In OCaml, the function
  \texttt{print\_string} is a simple way to print strings.

%\item Implement a suite of five \textbf{interesting} tests of stepping
%  once and multiple times.  (Do your expressions always have the same
%  type as they step?)  (``Interesting'' means that you attempt to
%  cover different execution paths of your implementation with each
%  test).
%
\end{enumerate}

\task{10} \paragraph{Continue reading the papers that you chose in Homework \#2.}

For each of the five papers, and for each question below, write two concise sentences:
\begin{enumerate}
\item Why did \emph{you} select this paper?
\item What is the ``main idea'' of the paper?
\item How well is this main idea communicated to you when you read the
  \emph{first two sections and conclusion} of paper, and skimmed the
  rest?  In particular, explain what aspects seem important, are which
  are clear versus unclear.  You may want to read deeper into the
  details of the paper body if these beginning and ending sections do
  not make the main ideas clear; make a note if this is required.
\end{enumerate}


\task{10} \paragraph{Continue thinking about your class project.}
%
Write an updated 250 word explanation of your plan, and what you hope
to accomplish with your project by the end of the semester.

That is, on what artifact do you want to be graded?  Recall that you
may choose to write a survey paper or implement something, but even
implementation projects require a short report.  By writing your plan
now, you are also generating a draft of part of this report.

Here are the same suggestions as on the prior homework:
\begin{itemize}
\item \textbf{Functional implementation:}
  Consider an algorithm or system that seems interesting to you.
  Can you write this algorithm in a purely-functional style in OCaml?
\item \textbf{Functional reactive implementation:} Learn a functional
  reactive programming language like Elm. (See
  \url{http://elm-lang.org/}). Write a game, simulation or
  productivity application in this new language.
\item \textbf{Survey project:} Choose a theme and six to eight papers
  from POPL, PLDI, ICFP and OOPSLA (or other ACM SIGPLAN Conferences
  in PL).  Write a survey paper about these papers, trying to tell a
  cohesive story about how they relate.
\end{itemize}

\end{document}
