\documentclass[11pt]{article}
\usepackage{fullpage}
\usepackage{latexsym}
\usepackage{verbatim}
\usepackage{code,proof,amsthm,amssymb, amsmath}
\usepackage{ifthen}
\usepackage{graphics}
\usepackage{xspace}
\usepackage{hyperref}
%\usepackage{mathpartir}

%% Question macros
\newcounter{question}[section]
\newcounter{extracredit}[section]
\newcounter{totalPoints}
\setcounter{totalPoints}{0}
\newcommand{\question}[1]
{
\bigskip
\addtocounter{question}{1}
\addtocounter{totalPoints}{#1}
\noindent
{\textbf{Task \thesection.\thequestion}}[#1 points]:


}
\newcommand{\ecquestion}
{
\bigskip
\addtocounter{extracredit}{1}
\noindent
\textbf{Extra Credit \thesection.\theextracredit}:}

%% Set to "false" to generate the problem set, set to "true" to generate the solution set
\def\issolution{false}

\newcounter{taskcounter}
\newcounter{taskPercentCounter}
\newcounter{taskcounterSection}
\setcounter{taskcounter}{1}
\setcounter{taskPercentCounter}{0}
\setcounter{taskcounterSection}{\value{section}}
\newcommand{\mayresettaskcounter}{\ifthenelse{\value{taskcounterSection} < \value{section}}
{\setcounter{taskcounterSection}{\value{section}}\setcounter{taskcounter}{1}}
{}}

% Solution-only - uses an "input" so that it's still safe to publish the problem set file
\definecolor{solutioncolor}{rgb}{0.0, 0.0, 0.5}
\newcommand{\solution}[1]
  {\ifthenelse{\equal{\issolution}{true}}
  {\begin{quote}
    \addtocounter{taskcounter}{-1}
    \fbox{\textcolor{solutioncolor}{\bf Solution \arabic{section}.\arabic{taskcounter}}}
    \addtocounter{taskcounter}{1}
    \textcolor{solutioncolor}{\input{./solution/#1}}
  \end{quote}}
  {}}

\newcommand{\qbox}{\fbox{???}}

\setcounter{taskcounter}{1}
\setcounter{taskPercentCounter}{0}
%\setcounter{taskcounterSection}{\value{section}}

\newcommand{\ttt}[1]{\texttt{#1}}

%% part of a problem
\newcommand{\task}[1]
  {\bigskip \noindent {\bf Task\addtocounter{taskPercentCounter}{#1} \arabic{taskcounter}\addtocounter{taskcounter}{1}} (#1 pts).}

\newcommand{\ectask}
  {\bigskip \noindent {\bf Task \arabic{section}.\arabic{taskcounter}\addtocounter{taskcounter}{1}} (Extra Credit).}

\newcommand{\val}[1]{#1~\textsf{val}}
\newcommand{\num}[1]{\texttt{num}[#1]}
\newcommand{\str}[1]{\texttt{str}[#1]}
\newcommand{\plus}[2]{\texttt{plus}(#1; #2)}
\newcommand{\mult}[2]{\texttt{times}(#1; #2)}
\newcommand{\cat}[2]{\texttt{cat}(#1; #2)}
\newcommand{\len}[1]{\texttt{len}(#1)}
\newcommand{\abst}[2]{#1.#2}
\newcommand{\letbind}[3]{\texttt{let}(#1; \abst{#2}{#3})}
\newcommand{\steps}[2]{#1 \mapsto #2}
\newcommand{\subst}[3]{[#1/#2]#3}
\newcommand{\err}[1]{#1~\textsf{err}}

\newcommand{\proves}{\vdash}
\newcommand{\hasType}[2]{#1 : #2}
\newcommand{\typeJ}[3]{#1 \proves \hasType{#2}{#3}}
\newcommand{\ctx}{\Gamma}
\newcommand{\emptyCtx}{\emptyset}
\newcommand{\xCtx}[2]{\ctx, \hasType{#1}{#2}}
\newcommand{\typeJC}[2]{\typeJ{\ctx}{#1}{#2}}
\newcommand{\numt}{\texttt{num}}
\newcommand{\strt}{\texttt{str}}

\newcommand{\laz}[1]{\left[ #1 \right]}

\newcommand{\E}{\textbf{\textsf{E}}\xspace}
\newcommand{\T}{\textbf{\textsf{T}}\xspace}

\title{Assignment \#2: \\
  %Abstract Binding Trees,
  Meta Theory and Implementation:
  \\Language~\T, with Finite Data Types}

\author{Fundamentals of Programming Languages}
% put together by Cyrus Omar and Shayak Sen, based on assignments from S10 and S09
\date{Out: Thursday, Sept 30th, 2016\\
      Due: Thursday, Oct 13th, 2016 11:59pm EST}

\begin{document}
\maketitle

The tasks in this homework ask you to prove (``meta theoretical'')
properties about the language~\T and about finite data types, defined
in Ch. 8--10 of \emph{PFPL}.  These properties are ``meta
theoretical'' in that they give a theory about the theory of the
language, and are true about \emph{all} well-typed programs, not
specific, individual programs.
%
This homework also asks you to program an implementation of these
languages in OCaml.
%
We did this together in class for the simpler Language~\E; use this
code and the video as a guide.

\paragraph{Grading criteria:} To receive full credit for any proof below, you must \emph{at least} do the following:
\begin{itemize}
\item At the beginning of your proof, specify over what structure or derivation you are performing induction (i.e., which structure's inductive principle are you using?)
\item In the inductive cases of the proof, specify how you are applying the inductive hypothesis, and what result it gives you.
\end{itemize}
\textbf{If you omit these steps and/or do not make them explicit, you
  will receive zero credit for your proof.}  If you attempt to do
these steps, but you make a mistake, you may still receive some partial credit, depending on your proof.

\paragraph{Hint:}
If you are unsure about how to structure these proofs to receive full credit, 
\textbf{please refer to the HW \#1 Solution} on Moodle as a reference and guide.

\paragraph{Note on omitting redundant proof cases:} In the proofs below, some cases are very
similar to other cases, e.g., the cases for \texttt{plus} and
\texttt{times} in the proofs below are likely to be analogous, in that
(nearly) the same proof steps are used in each.
%
When this happens, you can omit the redundant cases as follows: If you
do one case, say for \texttt{plus}, you may (optionally) write in the
other case for \texttt{times} that it is ``analogous to the case
above, for \texttt{plus}''.  You must make this omission explicit, to
show that you have thought about it.  Further, this shortcut is only
applicable when the cases really are analogous, and (nearly) the same
steps apply in the proof.  \textbf{When in doubt, do not omit the
  proof case.}

\section*{Tasks}

\task{20} Meta theory for Language~\T (See PFPL Chapter 9).
\begin{enumerate} 
\item State the substitution lemma for Language~\T.
\item State the canonical forms lemma for Language~\T.
\item State and prove progress for Language~\T.
\item State and prove preservation for Language~\T.
\end{enumerate}

\task{20} Meta theory for finite data types (See PFPL Chapters 10 and 11).
\begin{enumerate} 
\item State the substitution lemma for Language~\T extended with sum and product types.
\item State the canonical forms lemma for Language~\T extended with sum and product types.
\item State and prove progress for Language~\T extended with sum and product types.
\item State and prove preservation for Language~\T extended with sum and product types.
\end{enumerate}

\paragraph{\underline{Hint (repeated again, for emphasis)}:}
If you are unsure about how to structure these proofs to receive full credit, 
\textbf{please refer to the HW \#1 Solution} on Moodle as a reference and guide.

\task{25} Implement the theory of Language~\T in OCaml, along with tests.

\task{25} Extend the theory of Language~\T with Pairs and Sums.  Include additional tests.

\paragraph{How to implement a Language Theory:}
When we say ``implement Language $X$ in OCaml'', we mean precisely the
following.  For a concrete example, see our implementation of
Language~\E as a guide, where we did this together in class.  (There
is a lecture video and OCaml code from September 29 available online).
\begin{enumerate}
\item Define syntax forms as OCaml datatypes 
\begin{enumerate}
\item Define the syntax of expressions as a new OCaml datatype named~\texttt{exp}
\item Define the syntax of types as a new OCaml datatype named~\texttt{typ}
\item Define variables~\texttt{var} as OCaml strings (type~\texttt{string})
\item Define type contexts~\texttt{gamma} as OCaml lists of variable-type pairs.
\end{enumerate}
\item Implement a function \texttt{is\_val : exp -> bool} that implements a check for the $e~\textsf{val}$ judgment
\item Implement a substitution function \texttt{subst : exp -> var -> exp -> exp}.  
{
Make sure that you implement \emph{shadowing} correctly, and you do
not allow \emph{variable capture}.  
%
See our implemention of the \texttt{Let} case in language~\E
as a reference; notice how we compare the \texttt{Let}-bound variable
against the one being substituted, and do not substitute further if
they are the same.  }
\item Implement a type-checking function \texttt{exp\_typ : gamma ->
  exp -> typ option}.  

  \paragraph{Hint:}~Notice that to type-check lambda expressions \emph{without type
  annotations}, an implementation of type-checking must ``guess'' a
  type for the bound variable.  After all, this variable stands in for
  a parameter that we may not have locally; how would we know its type?

  This situation is different from the \texttt{Let} form in
  language~\E, where we have the sub-expression to which the
  variable is bound, and hence, we can get the type for the bound
  variable by processing this sub-expression.
  
  To avoid this guessing problem, define your syntax for lambda
  expressions to include a type for the argument variable.

\item Implement a suite of five \textbf{interesting} tests of type-checking.  
(``Interesting'' means that you attempt to cover different execution paths of your implementation with each test).
%

\item Implement a steps-to function \texttt{step : exp -> exp}. It
  should take exactly one small step, or raise an exception if the
  expression is a value.

\item Implement a multiple-steps function \texttt{steps : exp ->
  exp}. It should take as many steps as possible.  For programs that
  type-check, it should produce a value of the same type.  This is
  precisely what you proved in your \emph{meta theory} proofs about
  the language.

\item Implement a suite of five \textbf{interesting} tests of stepping once and multiple times.  
  (Do your expressions always have the same type as they step?)
(``Interesting'' means that you attempt to cover different execution paths of your implementation with each test).
%

\task{10} Start thinking about a class project.  Choose a partner, or
email the instructor if you would like to be (randomly assigned) with
others that want a random partner.  You may also work alone, if you
wish; people working in pairs must do more work than an individual
working alone.  Review the syllabus for detailed information about the
class project.

Recall, there are many possible class project options; here are a few ideas:

\begin{itemize}
\item \textbf{Functional implementation:}
  Consider an algorithm or system that seems interesting to you.
  Can you write this algorithm in a purely-functional style in OCaml?
\item \textbf{Functional reactive implementation:} Learn a functional
  reactive programming language like Elm. (See
  \url{http://elm-lang.org/}). Write a game, simulation or
  productivity application in this new language.
\item \textbf{Survey project:} Choose a theme and six to eight papers
  from POPL, PLDI, ICFP and OOPSLA (or other ACM SIGPLAN Conferences
  in PL).  Write a survey paper about these papers, trying to tell a
  cohesive story about how they relate.
\end{itemize}

To receive full points on this homework, you must do both of the following:

\begin{itemize}

\item Find a partner and discuss what project ideas seem most
  interesting to you both.
  %
If you already have a research project, consider how to incorporate it!
%
Write about your idea in 250 words.
%
Be clear about your interests, and what background reading you have done thus far.

\item Scan the titles of papers at (at least) five top PL conferences.
%
Name these conferences (including years), and include the conference
URL with this information.
%
For each conference, name the paper title and abstract that seems most
interesting to you from that conference's proceedings that year.
%
Include a URL for the author's draft of each paper, if any is available.
\end{itemize}

\end{enumerate}
\end{document}
