\documentclass{article}

\usepackage{mathpartir}
\usepackage{fullpage}

\newcommand{\arr}{\rightarrow}
\newcommand{\stepsto}{\longrightarrow}

\title{Final Exam\\
  CSCI 5535, Fall 2016}

\begin{document}
\maketitle

\section*{Tasks (100 Points)}

This exam consists of 100 points divided between OCaml-related tasks
and proof-related tasks below.
%
The OCaml tasks carry 25 points in total, whereas the proof tasks
carry 75 points in total.

Below these two sets of tasks, there are additional tasks that you
have the option of attempting to earn bonus points.  These bonus
tasks are optional; you need not attempt them to earn full credit on
the exam.

\subsection*{OCaml Tasks (25 Points, total)}~\\

\noindent
\textbf{5 Points:}\\
Define OCaml datatypes~\texttt{card} and~\texttt{cards} for the following inductive definitions:
\[\begin{array}{ccll}
\textrm{Card }  & c & ::= & \textrm{Heart} ~|~ \textrm{Diamond} ~|~ \textrm{Spade} ~|~ \textrm{Club}
\\
\textrm{Cards } & s & ::= & c :: s ~|~ \textrm{Empty}
\end{array}
\]

\paragraph{10 Points:}~\\
Define an OCaml function~\texttt{unshuffle} with the following type:
\begin{verbatim}
unshuffle : ('a -> card -> (bool * 'a))
            -> 'a -> cards -> (cards * cards)
\end{verbatim}
In particular, suppose a user runs your function as follows:
\begin{verbatim}
unshuffle choice init_state cards
\end{verbatim}
Where the user chooses expressions for \texttt{choice}, \texttt{init\_state} and \texttt{cards}.
\\[2mm]
\noindent
\textbf{Requirement 1:} When the choice
function~\texttt{choice} returns \texttt{true}, the algorithm for
\texttt{unshuffle} places the given card into the left deck of cards,
and when it returns \texttt{false}, it places the card into the right
deck.
\\[2mm]
\noindent
\textbf{Requirement 2:} Further, \texttt{unshuffle} should use the
choice state (of abstract type~\texttt{'a}) that results from each
choice to inform the next choice in the sequence of cards.
%
The initial choice state~\texttt{init\_state} is always given by the
user.

\paragraph{10 Points:}~\\
Define an OCaml function~\texttt{fivefive} with the following type:
\begin{verbatim}
fivefive : cards -> (cards * cards)
\end{verbatim}
This function should call \texttt{unshuffle} with a choice function
and initial choice state that ``unshuffles'' the cards from the input
list into two lists.
%
\\[2mm]
\noindent
\textbf{Requirement:} The first five cards from the input deck
should be placed in the left output deck.  The next five cards should
be placed into the right output deck.  Each following ten cards should
be distributed in the same way (five to the left, then five to the
right).  The number of cards may be zero or more, and need not be a
multiple of ten or five.  Until the cards end, \texttt{fivefive}
follows the pattern described above.

\section*{Proof Tasks (75 Points, total)}

Let us define a programming language~\textbf{ExamLang}, with the
following syntax, statics and dynamics (We covered this definition on
Thursday Dec 1, 2016):

\[\begin{array}{ccll}
\textrm{Expressions}  & e & ::= & \lambda x.e ~|~ e_1~e_2 ~|~ e_1 + e_2 ~|~ n ~|~ x
\\
\textrm{Types} & \tau & ::= & \texttt{num} ~|~ \tau_1 \arr \tau_2
\end{array}
\]

\fbox{$\Gamma \vdash e : \tau$} 
``Under $\Gamma$, expression~$e$ has type~$\tau$''
\begin{mathpar}
\inferrule*[right=S-var]
{ (x:\tau) \in \Gamma }
{ \Gamma \vdash x : \tau }
\and
\inferrule*[right=S-num]
{  }
{ \Gamma \vdash n : \texttt{num} }
\and
\inferrule*[right=S-plus]
{ \Gamma \vdash e_1 : \texttt{num}
  \\
  \Gamma \vdash e_2 : \texttt{num}
}
{ \Gamma \vdash e_1 + e_2 : \texttt{num} }
\and
\inferrule*[right=S-app]
{ 
  \Gamma \vdash e_1 : \tau_1 \arr \tau_2 
  \\
  \Gamma \vdash e_2 : \tau_1 
}
{ \Gamma \vdash e_1~e_2 : \tau_2 }
\and
\inferrule*[right=S-lam]
{ 
  \Gamma,~ x : \tau_1 \vdash e : \tau_2
}
{ \Gamma \vdash \lambda x. e : \tau_1 \arr \tau_2 }
\end{mathpar}


\fbox{$e~\textsf{val}$} 
``Expression~$e$ is a value''
\begin{mathpar}
\inferrule*[right=V-num]
{  }
{ n~~\textsf{val} }
\and
\inferrule*[right=V-lam]
{  }
{ \lambda x.e~~\textsf{val} }
\end{mathpar}

\fbox{$e \stepsto e'$} 
``expression~$e$ steps to expression~$e'$''
\begin{mathpar}
\inferrule*[right=D-App1]
{ e_1 \stepsto e_1' }
{ e_1~e_2 \stepsto e_1'~e_2}
\and
\inferrule*[right=D-App2]
{ e_2 \stepsto e_2' }
{ e_1~e_2 \stepsto e_1~e_2'}
\and
\inferrule*[right=D-App3]
{ }
{ (\lambda x.e_1)~e_2 \stepsto [e_2/x]e_1}
\and
\inferrule*[right=D-Plus1]
{ e_1 \stepsto e_1' }
{ e_1 + e_2 \stepsto e_1' + e_2}
\and
\inferrule*[right=D-Plus2]
{ e_2 \stepsto e_2' }
{ e_1 + e_2 \stepsto e_1 + e_2'}
\and
\inferrule*[right=D-Plus3]
{ n_1 + n_2 = n_3 }
{ n_1 + n_2 \stepsto n_3}
\end{mathpar}

\paragraph{Proof Task 1 (15 Points):}

\begin{itemize}
\item State (but do not prove) the substitution lemma for \textbf{ExamLang}
\item State (but do not prove) the canonical forms lemma for \textbf{ExamLang}
\item State the progress theorem for \textbf{ExamLang}. 
State over what structure to perform induction (but do not prove the cases).
\item State the preservation theorem for \textbf{ExamLang}.
State over what structure to perform induction (but do not prove the cases).
\item Consider the proofs for progress and preservation for \textbf{ExamLang}.  
We discussed these in class on Dec~1.  Answer these two questions:
\begin{itemize}
\item In which theorem, and in which proof case do we need to apply the substitution lemma?
\item In which theorem, and in which proof case do we need to apply the canonical forms lemma?
\end{itemize}
\end{itemize}

\paragraph{Proof Task 2 (25 Points):}

Suppose that we extend \textbf{ExamLang} with let-bound variables,
extending its abstract syntax with a new such form:

\[\begin{array}{ccll}
\textrm{Expressions}  & e & ::= & \cdots ~|~ \texttt{let}~x~\texttt{=}~e_1~\texttt{in}~e_2
\end{array}
\]

\paragraph{Subtasks:}
Your task is to update the definition and theorems for
\textbf{ExamLang} to account for this new language construct for~\texttt{let}.

\begin{itemize}
\item \textbf{Statics}: Extend the definition of $\Gamma \vdash e : \tau$ with one or more new rules.

\item \textbf{Dynamics}: Extend the definition of $e \stepsto e_2$ with one or more new rules.

\item \textbf{Canonical forms}: Does the canonical forms lemma (or its proof) need to
  change? If so, give the changes. If not, explain why not.

\item \textbf{Substitution}: Does the substitution lemma statement
  need to change? If so, give the changes. If not, explain why not.
  (Do not prove any new cases).

\item \textbf{Progress}: Give the \emph{new} case(s) of the proof.

\item \textbf{Preservation}: Give the \emph{new} case(s) of the proof.
\end{itemize}


\paragraph{Proof Task 3 (35 Points):}

Suppose that we further extend our language with lazy computations,
also known as thunks.
%
We extend the syntax with new forms, as follows:

\[\begin{array}{ccll}
\textrm{Expressions}  & e & ::= & \cdots ~|~ \texttt{thunk}~e ~|~ \texttt{force}~e
\\
\textrm{Types}     & \tau & ::= & \cdots ~|~ \textsf{thunk}(\tau)
\end{array}
\]

For some intuition, recall that in homework \#4, you use ``stream
thunks'' to represent the ``rest of the stream'' in the tail position
of each \texttt{Cons} cell.  Recall that in that OCaml code, we
represent these ``stream thunks'' as functions of type \texttt{unit ->
  stream}.  These functions accept the unit value as an argument and
they compute a stream value.  To ``force'' these thunks to compute, we
need only apply them to the unit value~\texttt{()} of unit type~\texttt{unit}.

More generally, thunks are values that represent suspended
computations.  When forced, thunks evaluate their suspended
computation and produce another value.

To make this informal intuition of thunks precise \emph{without
  encoding them as functions}, we extend the typing relation
(statics), value relation and stepping relation (dynamics) with the
following additional rules:
%
\\[2mm]
\fbox{$\Gamma \vdash e : \tau$} 
``Under $\Gamma$, expression~$e$ has type~$\tau$''
\begin{mathpar}
\inferrule*[right=S-thunk]
{ \Gamma \vdash e : \tau }
{ \Gamma \vdash \texttt{thunk}~e : \textsf{thunk}(\tau) }
\and
\inferrule*[right=S-force]
{ \Gamma \vdash e : \textsf{thunk}(\tau) }
{ \Gamma \vdash \texttt{force}~e : \tau }
\end{mathpar}


\fbox{$e~\textsf{val}$} 
``Expression~$e$ is a value''
\begin{mathpar}
\inferrule*[right=V-thunk]
{  }
{ \texttt{thunk}~e~~\textsf{val} }
\end{mathpar}

\fbox{$e \stepsto e'$} 
``expression~$e$ steps to expression~$e'$''
\begin{mathpar}
\inferrule*[right=D-force1]
{ e \stepsto e' }
{ \texttt{force}~e \stepsto \texttt{force}~e }
\and
\inferrule*[right=D-force2]
{ }
{ \texttt{force} (\texttt{thunk}~e) \stepsto e }
\end{mathpar}

\paragraph{Subtasks:}
Your task is to update our lemmas, theorems and proofs, by doing all of the following:

\begin{itemize}
%\item \textbf{Statics}: Extend the definition of $\Gamma \vdash e : \tau$ with one or more new rules.
%\item \textbf{Dynamics}: Extend the definition of $e \stepsto e_2$ with one or more new rules.

\item \textbf{Canonical forms}: Does the canonical forms lemma (or its proof) need to
  change? If so, give the changes. If not, explain why not.

\item \textbf{Substitution}: Does the substitution lemma statement
  \emph{or proof} need to change? If so, give the changes.
  (\emph{Please give proofs for any new cases}).  If not, explain why
  not.

\item \textbf{Progress}: Give the \emph{new} case(s) of the proof.

\item \textbf{Preservation}: Give the \emph{new} case(s) of the proof.
\end{itemize}

%\clearpage
\section*{Bonus Tasks (35 Points):}

\paragraph{Bonus Task 1 (15 Points):}
Characterize the asymptotic complexity of each of the following list
functions, as discussed in class on October 27 (see \texttt{ocaml0.ml}
on the lecture schedule).

For each, assume the input list(s) are of length $O(n)$.  Your
characterization should give the asymptotic time complexity, using
big-$O$ notation, for the number of reduction steps required to
evaluate each of the following functions on such input lists:
\begin{itemize}
\item \texttt{split}
\item \texttt{filter}
\item \texttt{sum}
\item \texttt{append} (assume both lists are of length $O(n)$)
\item \texttt{reverse} (the version that uses \texttt{append}, internally)
\item \texttt{reverse'} (the version that \emph{does not} use \texttt{append})
\item \texttt{BubbleSort.bubble} (the standard bubble-sort algorithm using lists)
\end{itemize}

\paragraph{Bonus Task 2 (20 Points):}
Characterize the asymptotic complexity of each of the following stream
functions, as described in Homework 4 (see \texttt{hw04.ml}), in the
\texttt{StreamNil} module (defining \emph{finite} streams).

For each function, assume the input streams(s) are finite, and of
length $O(n)$.  Your characterization should give the asymptotic time
complexity, using big-$O$ notation, for the number of reduction steps
required evaluate to each of the following functions on such input
lists.

\textbf{Important hint:} Unike lists, streams are lazy, since their
tails consist of stream \emph{thunks}.  
You should assume that creating a thunk
requires only $O(1)$ time, regardless of the thunk's suspended computation.

\begin{itemize}
\item \texttt{singletons}
\item \texttt{merge}
\item \texttt{merge\_adjacent}
\item \texttt{sort\_rec} 
\item \texttt{sort}
\end{itemize}


\end{document}
